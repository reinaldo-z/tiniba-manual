\subsection{a la Cabellos-Obsolete}


We leave this section just as DNA leaves all the usles junk that does
not codify proteins any more, and is expensive to throw it out, so
evolution's moto is if it ain't hurt juts carry it on. 
This subsection only runs $\chi_{ij}(\omega)$
and $\chi_{ijk}(2\omega)$ for bulk semiconductors. It assumes that the
experimental or {\it GW} band gap is known, like in GaAs.
We are planing to extend this version to other responses and the {\it
  layer-by-layer } analysis as well. $\chi_{ijk}(2\omega)$ is
calculated in the length-gauge  and in the velocity-gauge both wrongly
and correctly scissored. See Cabellos et al., Phys.  Rev. B {\bf 80}, 155205-1-13 (2009).

\begin{itemize}
\item go to:\\
\verb= cd $HOME/$TINIBA/SRC_response/SRC_set=  
\item Make a module. This has ``two'' names, one is the name of the
  file itself,
i.e. \verb=ishg1la.f90=
and the other is the name of the module itself,
i.e. \verb=IntegrandSHG1Mod.f90=,
 that goes as the first
line of the file.
The file with the module has two functions with the
response prefactor and a ``delta function factor'' depending if the
response is 1-$\omega$ or 2-$\omega$. It also has the subroutine with
the integrand that codifies the response to be calculated.\\
Then you have to link the new module to these next subroutines/modules: 

\item in \verb=IntegrandsMod.f90=
   \begin{itemize}
   \item at the top of the file insert the
     name of the new module, i.e. \\ \verb=USE IntegrandSHG1Mod.f90, ONLY :: SHG1= 
   \item Asign a case nomber, i.e.\\
     \verb=case (21)=\\
     \verb=CALL SGH1(i_spectra)=
   \end{itemize}
Notice that the case number is arbitrary but is is linked to the
\verb=ONLY :: SHG1= 

\item in \verb=SymmetryOperationsMod.f90=

\begin{itemize}

\item write the {\it case} number in the appropriate symmetry {\it transformation},
i.e. \\
\verb=  CASE(21,22,26,27,28,29,30,60,61,62,63,64,65,80,81)=\\
\verb=  CALL transformationSecondOrderResponse(i_spectra)=
     \end{itemize}
since the example is for SHG which is a second order response given by a third rank tensor.

\item in \verb=SpectrumParametersFileMod.f90=
      \begin{itemize}
      \item chose the tensor rank and its total number of Cartesian
        components, i.e.\\
\verb= CASE(21,22)= \\
\verb=  WRITE(6,*) "Second-harmonic generation: Length Gauge"=\\
\verb=  dims = = 3\\
\verb=  length = = 27
     \end{itemize}

since the rank is 3 and 3$^3$=27

\item in \verb=FileControlMod.f90= include

\begin{itemize}
\item \verb=USE IntegrandSHG1Mod, ONLY : SHG1_factor, SHG1DeltaFunctionFactor=
\item
\verb= CASE(21)=\\
\verb= rtmp2 = =\verb= rtmp*SHG1_factor()=\\
\verb= iTmp ==\verb= SHG1DeltaFunctionFactor()=

\end{itemize}


\end{itemize}        
When you finish, you have to compile in the \verb=master=, then you can make use of the 
advanced tool MakeMakefile.PL (no one uses it) :
\begin{enumerate}
\item \verb=$TINIBAU/=\textcolor{darkgreen}{MakeMakefile2010.PL}
\item
  \verb=$TINIBAU/=\textcolor{darkgreen}{compiler.sh}\\
and follow instructions$\ldots$ the executable is in the previous directory.
\end{enumerate}
% \begin{verbatim}
% Usage [Option]:
%  	3264        : Compile 32 and 64 bits(xeon,itanium and quad)
% 	xeon        : Compile 32  xeon
%  	itanium     : Compile 64 itanium
% 	quad        : Compile 64 quad
% 	example: ./compiler.sh 3264
% 	Stoping right now ...
% \end{verbatim}   



After this a script has to driver this subrutine, there are several of them 
if you want to use one of this:\\
You have to include the number of response in those scripts\\
/home/\$USER/\$TINIBA/SRC$_{-}$response/\textcolor{darkgreen}{responseSHG.sh}\\
In this script include a line like this
\begin{verbatim}
NAMERES[1]=chi1  ;SIZERES[1]="2"
\end{verbatim}
As you see one of this is the name of the subroutine and other is the lenght of the tensor\\ 
\verb=$HOME/$TINIBA/SRC_response/menu.sh= 
In this script include a line like this
\begin{verbatim}
printf "\t${GREEN}22${NC} ${BLU}${NAMERES[22]}${NC}  
(${GREEN}Length 2 omega${NC})\n"
\end{verbatim}

In order to be able to run, try it with small case. 
if you are not able to run ask to Bernardo en the coffe 
break in order to solve the problem. 
