\chapter{Optics}
 
The program to run all the optical responses is\\
$\bullet$\verb=$TINIBA/utils/all_responses.sh=\\ 
which in turn runs\\
$\bullet$\verb=$TINIBA/utils/responses.sh=\\
To run it just execute\\
$\bullet$ \verb=$PWD > all_responses.sh= \\
and follow instructions. In particular one can calculate from all
valence bands to any number of conductions bands (\verb=-o 1=),
or from a given valence band to a given conduction band (\verb=-o 2=).

\begin{itemize}
\item For SHG bulk/surface use response 21/44. Both are coded in the
  Length Gauge. Response 42 only works
  for bulk and is coded the
  Velocity Gauge. Since it has to calculate extra commutators is
  slower than response 21. Thus use 42 only if you doubt your results
  and would like to get some reasurance that they are correctly
  calculated by 21.

\item The file for SHG has for columns\\\\
\begin{tabular}{|c|c|c|c|c|c|}\hline 
Column $\to$  & 1 & 2 & 3 & 4 & 5 \\\hline 
Quantity $\to$  & $\hbar\go$  & $\mathrm{Re}[\chi_{ijk}(1\go)]$ &
$\mathrm{Im}[\chi_{ijk}(1\go)]$ &
$\mathrm{Re}[\chi_{ijk}(2\go)]$ &
$\mathrm{Im}[\chi_{ijk}(2\go)]$\\\hline
\end{tabular} 
Units: pm/V\\
\textcolor{red}{So far it only works for one component of $\chi_{ijk}$.}

\item Degree of Spin Polarization (DSP), $\cald^a$.
\begin{enumerate}
\item \textcolor{red}{WARNING}: the factor of $(\hbar/2)$ is NOT
  included in $\zeta^{abc}$. However it is included in the expression
  for $\cald^a$, and thus the value of $\zeta^{abc}$ as it comes out
  of the code must be used to compute $\cald^a$. To report the value
  of $\zeta^{abc}$ one must multiply by $\hbar/2$ and use the
  appropriate S.I. units V$^{-2}$m$^{-1}$J.
\item For a surface calculation use a symmetric slab, i.e. same top
  and bottom surfaces.
\item Use \verb=dsp.sh= to gather the results for the DSP calculation. It
  does so for bulk (full) or layer-by-layer. The data is written as
follows:\\\\
\begin{tabular}{|c|c|c|c|c|c|}\hline
Column $\to$  & 1 & 2 & 3 & 4 &5 \\\hline
\verb=kk=$\to$  & $\hbar\go$  & $\xi_{xx}$  &
$\xi_{yy}$ &$\xi_{zz}$ & $\zeta_{ijk}$ \\ \hline
\end{tabular}\\
\item \verb=dsp.sh= uses the \verb=kk= response files and not the \verb=sm=
files. The onset of the DSP is very sharp and the \verb=sm= files
misscalculate it.
\item\verb=gnuplot> p 'dsp-file' u 1:(2*$5/($j+$k+eta)) w l=\\
where $j$ and $k$ are the corresponding columns (2,3 or 4) according to the $jk$
Cartesian components of $\zeta_{ijk}$. Also,
\verb=eta->0=, so {\it gnuplot} plots the DSP onset properly.
\item If you have calculated \verb=dsp.sh= with the \verb=full= option
  for a surface, i.e. the whole ``bulk-like''
 structure,
 then you must get the same result as with the
  \verb=half-slab= option, since both $\xi^{ab}$ and $\zeta^{abc}$ are
  twice as much for the whole slab than for the half-slab, but the
  factor of two cancels out when taking the ratio in the expression for $\cald^a$.
\end{enumerate}

\end{itemize}

\section{Formulas and Units}


These are the formulas coded:

\begin{itemize}

\item Linear response (See Sipe and Shkrebtii PRB {\bf 61}, 5337 (2000) Eq. 34
% \begin{equation*}\label{chi}
% \mathrm{Im}[\chi^{ij}(\go)]=\frac{e^2\pi}{\hbar}
% \int \frac{d\bfk}{8\pi^3}
% \sum_{nm} 
% f_{nm} 
% r^i_{nm}(\bfk)r^j_{mn}(\bfk)\gd(\go_{mn}(\bfk)-\go)
% ,
% \end{equation*}  
where $m=c$ and $n=v$ for the resonant condition with $\go>0$),
\begin{equation*}\label{chif}
\mathrm{Im}[\chi^{\rma\rmb}(\go)]=\frac{\pi e^2}{\hbar}
\int \frac{d\bfk}{8\pi^3}
\sum_{vc}
r^{\rma}_{vc}(\bfk)r^{\rmb}_{cv}(\bfk)\gd(\go_{cv}(\bfk)-\go)
.
\end{equation*} 
For the layer response we replace $r^{\rma}_{vc}(\bfk)\to{\cal
 R}^{(\ell)\rma}_{vc}(\bfk)$ see Mendoza et al, Phys. Rev. B {\bf 74},
075318 (2006).

\item Carrier injection rate, 
$\dot n(\go)=\xi^{ij}(\go)E^i(-\go)E^j(\go)$,
where is better to redefine as $\dot{\tilde n}=(\hbar/2)\dot n=\tilde\xi^{ab}E^a(-\go)E^b(\go)$, with
$\tilde\xi^{ab}=(\hbar/2)\xi^{ab}$.
\begin{equation*}\label{chifp}
\tilde\xi^{\rma\rmb}(\go)=\frac{\pi e^2}{\hbar}
\int \frac{d\bfk}{8\pi^3}
\sum_{vc}
r^{\rma}_{vc}(\bfk)r^{\rmb}_{cv}(\bfk)\gd(\go_{cv}(\bfk)-\go)
.
\end{equation*} 
For the layer response:
\begin{equation*}\label{xizn}
\tilde\xi^{ab}(\ell;\go)
=
\frac{\pi e^2}{\hbar}
\int\frac{d^3k}{8\pi^3}
\sum_{vcc'}
\frac{1}{2}\mathrm{Re}\Big[
\rho_{cc'}(\ell)    
r^a_{vc} 
r^b_{c'v}
+
\rho_{c'c}(\ell) 
r^a_{vc'} 
r^b_{cv}
\Big]
\gd(\omega-\omega_{cv})
.
\end{equation*}
We notice that $\tilde\xi^{ab}(\go)=\mathrm{Im}[\chi^{ab}(\go)]$, with
$\ge^{ab}(\go)=1+4\pi\chi^{ab}(\go)$, however 
$\tilde\xi^{ab}(\ell;\go)\neq\mathrm{Im}[\chi^{ab}(\ell;\go)]$!!!
Therefore, for a \textcolor{red}{bulk} calculation, one must calculate
$\mathrm{Im}[\chi^{ab}(\go)]$ for $\dot n(\go)$.

\item Second Harmonic Generation
\begin{itemize}
\item Velocity-Gauge 

The SHG $\chi_{\rmv}^{\rma\rmb\rmc}(2\go)$ is programmed within the velocity-gauge
 according to Phys.  Rev. B {\bf 80}, 155205-1-13 (2009).
\begin{eqnarray*}
&&\mbox{Im}\lbrack \chi _{\mathrm{v}}^{abc}(-2\omega;\omega,\omega)] 
=\frac{\pi
|e|^{3}}{2\hbar ^{2}}\int \frac{d^{3}k}{8\pi ^{3}}\Big[\sum_{vc}\frac{16}{(
\omega_{cv}^{S})^{3}}\Big(\sum_{c^{\prime }}\frac{\mbox{Im}\lbrack
v_{vc}^{\Sigma ,a}\{v_{cc^{\prime }}^{\Sigma ,b}v_{c^{\prime }v}^{\Sigma
,c}\}]}{\omega_{cv}^{S}-2\omega_{c^{\prime }v}^{S}}  \notag  \label{imchicf} \\
&-&\sum_{v^{\prime }}\frac{\mbox{Im}\lbrack v_{vc}^{\Sigma
,a}\{v_{cv^{\prime }}^{\Sigma ,b}v_{v^{\prime }v}^{\Sigma ,c}\}]}{\omega
_{cv}^{S}-2\omega_{cv^{\prime }}^{S}}\Big)\delta(\omega_{cv}^{S}-2\omega)  \notag \\
&+&\sum_{(vc)\neq\ell }\frac{1}{(\omega_{cv}^{S})^{3}}\Big(\frac{\mbox{Im}\lbrack
v_{\ell c}^{\Sigma ,a}\{v_{cv}^{\Sigma ,b}v_{v\ell }^{\Sigma ,c}\}]}{\omega
_{c\ell }^{S}-2\omega_{cv}^{S}}-\frac{\mbox{Im}\lbrack v_{v\ell }^{\Sigma
,a}\{v_{\ell c}^{\Sigma ,b}v_{cv}^{\Sigma ,c}\}]}{\omega_{\ell v}^{S}-2\omega
_{cv}^{S}}\Big)\delta(\omega_{cv}^{S}-\omega)  \notag \\
&-&\sum_{vc}\frac{1}{(\omega_{cv}^{S})^{3}}\Big(4\mbox{Re}\lbrack \
v_{vc}^{\Sigma ,a}\{\mathcal{F}_{cv}^{bc}\}]\delta(\omega_{cv}^{S}-2\omega)+\mbox{Re}
\lbrack \{\mathcal{F}_{vc}^{ab}v_{cv}^{\Sigma ,c}\}]\delta(\omega_{cv}^{S}-\omega)
\Big)\Big].
\end{eqnarray*}
Programs: \verb=shg1v= and \verb=shg2v=

\item Length-Gauge: 
Layered response by bms-unpublished. See \verb=shg-layer.pdf=
\begin{equation*}\label{imchiewn}
\mathrm{Im}[\chi_{e,\rma\rmb\rmc,\go}^{s(\ell)}]
=
\frac{\pi |e|^3}{2\hbar^2} 
\sum_{vc\bfk}
\sum_{l\neq(v,c)}
\left[
\frac{\go^S_{lc}\mathrm{Re}[{\cal R}^{\rma(\ell)}_{lc}\{r^{\rmb}_{cv}r^{\rmc}_{vl}\}]}
{\go^S_{cv}(2\go^S_{cv}-\go^S_{cl})}
-
\frac{\go^S_{vl}\mathrm{Re}[{\cal R}^{\rma(\ell)}_{vl}\{r^{\rmc}_{lc}r^{\rmb}_{cv}\}]}
{\go^S_{cv}(2\go^S_{cv}-\go^S_{lv})}
\right]
\gd(\go^S_{cv}-\go)
\end{equation*}  
\begin{equation*}\label{imchiwn}
\mathrm{Im}[\chi_{i,\rma\rmb\rmc,\go}^{s(\ell)}]
=
\frac{\pi|e|^3}{2\hbar^2}
\sum_{cv\bfk}
\frac{1}{\go^S_{cv}}
\left[
\mathrm{Im}[\{r^{\rmb}_{cv}\left({\cal R}^{\rma(\ell)}_{vc}\right)_{;k^{\rmc}}\}]
+
\frac{2\mathrm{Im}[{\cal R}^{\rma(\ell)}_{vc}\{r^{\rmb}_{cv}\gD^{\rmc}_{cv}\}]}{\go^S_{cv}}
\right]
\gd(\go^S_{cv}-\go)
\end{equation*}
\begin{equation*}\label{imchie2wn}
\mathrm{Im}[\chi_{e,\rma\rmb\rmc,2\go}^{s(\ell)}]
=
\frac{\pi |e|^3}{2\hbar^2} 
\sum_{vc\bfk}
4
\left[
\sum_{v'\ne v}
\frac{\mathrm{Re}[{\cal
    R}^{\rma(\ell)}_{vc}\{r^{\rmb}_{cv'}r^{\rmc}_{v'v}\}]}{2\go^S_{cv'}-\go^S_{cv}}
-
\sum_{c'\ne c}
\frac{\mathrm{Re}[{\cal R}^{\rma(\ell)}_{vc}\{r^{\rmc}_{cc'}r^{\rmb}_{c'v}\}]}
{2\go^S_{c'v}-\go^S_{cv}}
\right]
\gd(\go^S_{cv}-2\go)
\end{equation*}
\begin{equation*}\label{imchi2wn}
\mathrm{Im}[\chi_{i,\rma\rmb\rmc,2\go}^{s(\ell)}]
=
\frac{\pi|e|^3}{2\hbar^2}\sum_{vc\bfk}
\frac{4}{\go^S_{cv}}
\left[
\mathrm{Im}[{\cal R}^{\rma(\ell)}_{vc}\{\left(r^{\rmb}_{cv}\right)_{;k^{\rmc}}\}]
-
\frac{2\mathrm{Im}[{\cal R}^{\rma(\ell)}_{vc}\{r^{\rmb}_{cv}\gD^{\rmc}_{cv}\}]}{\go^S_{cv}}
\right]\gd(\go^S_{cv}-2\go)
\end{equation*}
\end{itemize}
Programs: \verb=shg1l= and \verb=shg2l= for bulk, i.e. ${\cal
  R}^{\rma(\ell)}_{vc}\to r^\rma_{vc}$.\\
\verb=shg1c= and \verb=shg2c= for layered.

\item Injection current

\begin{equation*}\label{zetaell}
\eta^{\rma\rmb\rmc}(\ell|0;\go,-\go)
=
\frac{i\pi e^3}{\hbar^2}
\intk
\sum_{vc}
\gD_{cv}^{\rma}(\ell;\bfk)
\mathrm{Im}
\big[ 
r^{\rmb}_{cv}(\bfk) 
r^{\rmc}_{vc}(\bfk)
\big]
\gd(\go_{cv}(\bfk)-\go)
.\nonumber
\end{equation*} 

For the bulk response we replace 
$\gD_{cv}^{\rma}(\ell;\bfk) \to \gD_{cv}^{\rma}(\bfk)$

\item Spin injection

\begin{eqnarray*}\label{zetaabci}
\zeta^{\mathrm{abc}}(\omega)
&=&
\frac{i\pi e^2}{\hbar^2}
\int\frac{d^3k}{8\pi^3}
\sum_{vcc'}\,'\,
\mathrm{Im}\Big[S^{\mathrm{a}}_{c'c}(\mathbf{k}) r^{\mathrm{b}}_{vc'}(\mathbf{k}) r^{\mathrm{c}}_{cv}(\mathbf{k})
\nonumber\\
&+& 
S^{\mathrm{a}}_{cc'}(\mathbf{k}) r^{\mathrm{b}}_{vc}(\mathbf{k}) r^{\mathrm{c}}_{c'v}(\mathbf{k})\Big]
\delta(\omega_{cv}(\mathbf{k})-\omega)
.
\end{eqnarray*}  
Notice that the units of $\zeta^{\rma\rmb\rmc}(\go)$ have a $\hbar/2$
factor coming from the spin matrix elements $S^{\rma}_{cc'}$ (recall
that
$\hat \bfS=(\hbar/2)\hat\bfgs$), besides the other units.
The degree of spin polarization is defined as
\begin{equation}\label{dps}
{\cal
  D}=\frac{2\zeta^{\rmz\rmx\rmy}}{\hbar(\xi^{\rmx\rmx}+\xi^{\rmy\rmy})/2}
=\frac{2\zeta^{\rmz\rmx\rmy}}{(\tilde\xi^{\rmx\rmx}+\tilde\xi^{\rmy\rmy})}
,\nonumber
\end{equation}    
and it is a dimensionless quantity, as it must. 

For the layered response we replace $S^{\rma}_{cc'}(\bfk)\to {\cal S}^{\ell,\rma}_{cc'}(\bfk)$. 

\item Units. 
\begin{itemize}

\item In general, 
\begin{equation*}\label{si}
\ge_0\chi_{\mathrm{S.I.}}^{j}=\frac{4\pi\ge_0}{(3\times
 10^4)^{j-1}}\chi^{j}_{\mathrm{c.g.s}} \times \frac{m^{j-2}C}{V^j}
,
\end{equation*} 
with $j$ the order of the response. 

\item As an example we work the injection current units:

\begin{equation*}\label{unp}
\gamma=\frac{\pi e^3}{\hbar^2}\times\frac{1}{\gO}\times v \times 
r\times r \times \frac{1}{\go}
,
\end{equation*} 
the first term is the prefactor, the second is the volume of the
$\bfk$-integration, the third is the velocity of $\gD$ the fourth and
fifth are the matrix elements of the position operator and the last
one comes from de Dirac delta function. Using $r=v/\go$ unit-wise
\begin{eqnarray*}\label{un1}
\gamma
&=&
\frac{\pi e^3}{\hbar^2}\times\frac{1}{a_0^3}\times v \times
\frac{v^2}{\go^2}
\times \frac{1}{\go}
\nonumber\\
&=&
\frac{\pi e^3}{\hbar^2a_0^3}\times
\frac{v^3}{\go^3}
\nonumber\\
&=&
\frac{\pi e^3\hbar}{a_0^3}\times
\frac{p^3}{m^3(\hbar\go)^3}
=
\frac{\pi e^3\hbar}{m^3a_0^3}\times
\frac{\hbar^3}{a_0^3}
\times
\frac{1}{[\mathrm{eV}]^3}
\nonumber\\
&=&
\frac{\pi e^3\hbar^4}{m^3a_0^6}\times
\frac{1}{[\mathrm{eV}]^3}
\times
\frac{[27.21\,\mathrm{eV}]^3}{H^3}
\nonumber\\
&=&
\frac{\pi e^3\hbar^4}{m^3a_0^6}\times
[27.21]^3
\times
\frac{a_0^3}{e^6}
=
\frac{\pi \hbar^4}{e^3(ma_0)^3}\times
[27.21]^3
\nonumber\\
&=&
\frac{\pi \hbar^4}{e^3(\hbar^2/e^2)^3}\times
[27.21]^3
=
\frac{\pi e^3}{\hbar^2}\times
[27.21]^3
,
\end{eqnarray*}
with $a_0=\hbar^2/me^2$ Bohr's radius, $\hbar\go$ is measured in eV, $v=p/m$ and
$[p]=\hbar/a_0$, the Hartree $H=e^2/a_0=27.21$ eV, in c.g.s
$e=-4.8066\times 10^{-10}$ statcoulomb and $\hbar=1.05457\times
10^{-27}$ erg$\cdot$s, then
\begin{equation*}\label{gam2}
\gamma=-
\pi
\times
[27.21]^3
\times
[9.9853\times 10^{25}]
.
\end{equation*}

For the injection current we
have $j=2$, 
\begin{equation*}\label{si2}
\eta_{\mathrm{S.I.}}=\frac{4\pi\ge_0}{3\times 10^4}\eta_{\mathrm{c.g.s}} \times \frac{C}{V^2}
,
\end{equation*} 
and
\begin{equation*}\label{is}
\frac{1}{4\pi\ge_0}=9\times 10^9%\frac{Nm^2}{C^2}
,
\end{equation*} 
since the units of $\ge_0$ are already taken into account,
then
\begin{equation*}\label{si3}
\eta_{\mathrm{S.I.}}=\frac{1}{27\times 10^{13}}\eta_{\mathrm{c.g.s}} \times \frac{C^3}{J^2}
,
\end{equation*}
where $J=VC$. However, $\eta$ gives the injection current,
i.e. $dJ/dt=\eta E E$, then there is a second coming from the $J$ and
another second coming from the time derivative, which finally give
\begin{eqnarray*}\label{gam}
\gamma
&=&
-\pi
\times
[27.21]^3
\times
[9.9853\times 10^{25}]
\times\frac{1}{27\times 10^{13}}
\nonumber\\
&=&
-\pi
\times
[27.21]^3
\times
[9.9853\times 10^{25}]
\times[3.7037\times 10^{-15}]
,
\end{eqnarray*}
as the prefactor of $\eta$ in $C^3/J^2 s^2$. For the surface $\eta_s$
there is a factor of meters since $\eta_s\sim L \eta$, then the units
of $\eta_s$ are $mC^3/J^2s^2$. The $L$ factor is set at the
\verb=gnuplot= file for the plot.

We remark that for the injection current the bulk prefactor
$\gamma_{\mathrm{bulk}}=\gamma/2$ since we use the commutator instead
of the imaginary part of $ r^{\rmb}_{cv} r^{\rmc}_{vc}$ and
$i\mathrm{Im}[ab]=[a,b]/2$.

\item SHG
\begin{equation*}\label{shgu}
[\chi]=\frac{\pi}{2}\frac{[e]^3}{[\gO][\hbar]^2}\frac{[r]^3}{[\go]^2}
,
\end{equation*}

    % volume = BohrRadius_cgs ** 3
    % tmp1 = (electronCharge_cgs ** 3) / (hbar_cgs ** 2)
    % tmp1 = tmp1 * (positionFactor_cgs ** 3)
    % tmp1 = tmp1 / (frequency_cgs ** 2)
    % tmp1 = tmp1 / volume
    % res = makeDouble(tmp1)
    % SHG1_factor = 2.*res*pi 
    % !######### 29 Nov. 2008 ##########
    % !right now in pmV
    % !##pag. 53 Nonlinear Optics Robert Boyd
    % !K=4.189E-4*1E12
    %  SHG1_factor= SHG1_factor*4.189E-4*1E12
    %  SHG1_factor= SHG1_factor/2.

\item Values: 
Are given in \verb=cgs= and \verb=mks= units in\\
 \verb=~/tiniba/tiniba2010/SRC_response/SRC_set/PhysicalConstantsMod.f90=
% the
%  length-gauge can be found in:\\
% \verb=/home/bms/tiniba/tiniba2010/SRC_response/SRC_set/=\\
% \verb=ishg1la.f90= for 1-$\omega$ and
% \verb=ishg2la.f90= for 2-$\omega$.

   \end{itemize}

\end{itemize}
