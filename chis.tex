Using 
$\mathrm{Re}[iz]=-\mathrm{Im}[z]$,
$\mathrm{Im}[iz]=\mathrm{Re}[z]$,
and
\begin{equation}\label{rnmenm91}
{\cal R}^\rma_{nm}=\frac{{\cal P}^\rma_{nm}}{im_e\go_{nm}}
=\frac{\calv^{LDA,\rma}_{nm}}{i\go^{LDA}_{nm}}
=\frac{\calv^{\gS,\rma}_{nm}}{i\go^\gS_{nm}}
\quad\quad n\ne m  
,
\end{equation} 
and
\begin{equation}\label{rnmenm92n}
{\cal R}^\rma_{nm;k^b}=\left(\frac{{\cal V}^{\gS,a}_{nm}}{i\go^\gS_{nm}}\right)_{;k^b}
=\frac{{\cal V}^{\gS,a}_{nm;k^b}}{i\go^\gS_{nm}}
-
\frac{{\cal V}^{\gS,a}_{nm}\gD^b_{nm}}{i(\go^\gS_{nm})^2}
\quad\quad n\ne m 
,
\end{equation}
we can show the equivalence between the two formulations.
As of May 27, 2016, we take $\gs\to\gS$, $S\to\gS$, remove the $\ell$
to denote layer, since the calligraphic type implicitly denotes it,
the $s$ from $\chi^{\rma\rmb\rmc}$, understanding that depending on whether is a
surface or a bulk calculation, $\chi^{\rma\rmb\rmc}$ is a surface or a
bulk quantity.  
\begin{align}\label{imchiew}
\mathrm{Im}[\chi_{e,\go}^{\rma\rmb\rmc}]
&=
\frac{\pi |e|^3}{2\hbar^2} 
\sum_{vc\bfk}
\sum_{l\neq(v,c)}
\left[
\frac{\go^\gS_{lc}\mathrm{Re}[{\cal R}^{\rma}_{lc}\{r^{\rmb}_{cv}r^{\rmc}_{vl}\}]}
{\go^\gS_{cv}(2\go^\gS_{cv}-\go^\gS_{cl})}
-
\frac{\go^\gS_{vl}\mathrm{Re}[{\cal R}^{\rma}_{vl}\{r^{\rmc}_{lc}r^{\rmb}_{cv}\}]}
{\go^\gS_{cv}(2\go^\gS_{cv}-\go^\gS_{lv})}
\right]
\gd(\go^\gS_{cv}-\go)
,
\end{align}  
\begin{equation}\label{c-calvimchiewn}
\mathrm{Im}[\chi_{e,\go}^{\rma\rmb\rmc}] =
\frac{\pi |e|^3}{2\hbar^2}\sum_{vc\bfk}\sum_{l\neq(v,c)}\frac{1}{\omega^\gS_{cv}}
\left[
\frac{\mathrm{Im}[\mathcal{V}^{\gS,\text{a}}_{lc}\{r^{\rmb}_{cv}r^{\rmc}_{vl}\}]}
{(2\go^\gS_{cv}-\go^\gS_{cl})} 
-\frac{\mathrm{Im}[\mathcal{V}^{\gS,\text{a}}_{vl}\{r^{\rmc}_{lc}r^{\rmb}_{cv}\}]}
{(2\go^\gS_{cv}-\go^\gS_{lv})}
\right]\gd(\go^\gS_{cv}-\go),
\end{equation}  
\begin{align}\label{imchiwf}
\mathrm{Im}[\chi_{i,\go}^{\rma\rmb\rmc}]
&=
\frac{\pi|e|^3}{2\hbar^2}
\sum_{cv\bfk}
\frac{1}{\go^\gS_{cv}}
\left[
\mathrm{Im}[\{r^{\rmb}_{cv}\left({\cal R}^{\rma}_{vc}\right)_{;k^{\rmc}}\}]
+
\frac{2\mathrm{Im}[{\cal R}^{\rma}_{vc}\{r^{\rmb}_{cv}\gD^{\rmc}_{cv}\}]}{\go^\gS_{cv}}
\right]
\gd(\go^\gS_{cv}-\go)
,
\end{align}
\begin{equation}\label{c-calvimchiwn}
\mathrm{Im}[\chi_{i,\omega}^{\rma\rmb\rmc}]
= \frac{\pi\vert e\vert^3}{2\hbar^2}\sum_{cv\mathbf{k}}\frac{1}{(\omega^\gS_{cv})^{2}}
\left[
\mathrm{Re}\left[\left\{r^{\text{b}}_{cv}\left(\mathcal{V}^{\gS,\text{a}}_{vc}\right)_{;k^{\text{c}}}\right\}\right]
+\frac{\mathrm{Re}\left[\mathcal{V}^{\gS,\text{a}}_{vc}\left\{r^{\text{b}}_{cv}
\Delta^{\text{c}}_{cv}\right\}\right]}{\omega^\gS_{cv}} 
\right]\delta(\omega^\gS_{cv}-\omega),
\end{equation}
\begin{align}\label{imchie2w}
\mathrm{Im}[\chi_{e,2\go}^{\rma\rmb\rmc}]
&=
\frac{\pi |e|^3}{2\hbar^2} 
\sum_{vc\bfk} 
4
\left[
\sum_{v'\ne v}
\frac{\mathrm{Re}[{\cal
    R}^{\rma}_{vc}\{r^{\rmb}_{cv'}r^{\rmc}_{v'v}\}]}{2\go^\gS_{cv'}-\go^\gS_{cv}}
-
\sum_{c'\ne c}
\frac{\mathrm{Re}[{\cal R}^{\rma}_{vc}\{r^{\rmc}_{cc'}r^{\rmb}_{c'v}\}]}
{2\go^\gS_{c'v}-\go^\gS_{cv}}
\right]
\gd(\go^\gS_{cv}-2\go)
,
\end{align}  
\begin{equation}\label{c-calvimchie2wn}
\mathrm{Im}[\chi_{e,2\go}^{\rma\rmb\rmc}] =
-\frac{\pi |e|^3}{2\hbar^2}\sum_{vc\bfk}\frac{4}{\omega^\gS_{cv}}
\left[
\sum_{v'\ne
  v}\frac{\mathrm{Im}[\mathcal{V}^{\gS,\text{a}}_{vc}\{r^{\rmb}_{cv'}r^{\rmc}_{v'v}\}]}
{2\go^\gS_{cv'}-\go^\gS_{cv}}
- \sum_{c'\ne
  c}\frac{\mathrm{Im}[\mathcal{V}^{\gS,\text{a}}_{vc}\{r^{\rmc}_{cc'}r^{\rmb}_{c'v}\}]}
{2\go^\gS_{c'v}-\go^\gS_{cv}}
\right]\gd(\go^\gS_{cv}-2\go),
\end{equation}
 and
\begin{align}\label{imchi2wf}
\mathrm{Im}[\chi_{i,2\go}^{\rma\rmb\rmc}]
&=
\frac{\pi|e|^3}{2\hbar^2}\sum_{vc\bfk}
\frac{4}{\go^\gS_{cv}}
\left[
\mathrm{Im}[{\cal R}^{\rma}_{vc}\{\left(r^{\rmb}_{cv}\right)_{;k^{\rmc}}\}]
-
\frac{2\mathrm{Im}[{\cal R}^{\rma}_{vc}\{r^{\rmb}_{cv}\gD^{\rmc}_{cv}\}]}{\go^\gS_{cv}}
\right]\gd(\go^\gS_{cv}-2\go)
,
\end{align}
\begin{equation}\label{c-calvimchi2wn}
\mathrm{Im}[\chi_{i,2\omega}^{\rma\rmb\rmc}] 
=
 \frac{\pi \vert
   e\vert^{3}}{2\hbar^2}\sum_{vc\mathbf{k}}\frac{4}{(\omega^\gS_{cv})^{2}}
\left[\mathrm{Re}\left[\mathcal{V}^{\gS,\text{a}}_{vc}\left\{\left(r^{\text{b}}_{cv}\right)_{;k^{\text{c}}}
\right\}\right] -
\frac{2\mathrm{Re}\left[\mathcal{V}^{\gS,\text{a}}_{vc}\left\{r^{\text{b}}_{cv}
\Delta^{\text{c}}_{cv}\right\}\right]}{\omega^\gS_{cv}}\right]\delta(\omega^\gS_{cv}-2\omega)
,
\end{equation}


\noindent If we take ${\cal R}^{\rma}_{nm}\to r^\rma_{nm}$, 
or
${\cal V}^{\gS,a}_{nm}\to v^{\gS,\rma}_{nm}$, 
we
would recover the expressions for a bulk response.
We prefer to display the expressions in terms of $\calv^{\gS,a}_{nm}$, since they are
more physically appealing, as the velocity is what gives the current
of a given layer, from which the polarization is computed and the
$\chi^{\rma\rmb\rmc}$ is extracted. However in \tiniba, the
expressions with $\calr^\rma_{nm}$ and     
$\calr^\rma_{nm;k^\rmb}$, are the ones that are coded, i.e.
Eq.~\eqref{imchiew}, \eqref{imchiwf}, \eqref{imchie2w}, 
and \eqref{imchi2wf}.

\textcolor{red}{Remark:}  
The code normalizes $\chi^{\text{code}}$ by the volume $V$ of the unit cell, i.e.
$\chi^{\text{code}}=\tilde\chi^{\text{code}}/V$, where
$\tilde\chi^{\text{code}}$ is the unnormalized output value of the code.
 For a
surface calculation, the volume of the supercell is $V=AL$ where $L$
is the height of the supercell, denoted by $L=$\verb=acellz= in
\verb=TINIBA=. Therefore, to obtain the surface
$\chi^{\text{surface}}$ we take
\begin{equation}\label{chisurf}
\chi^{\text{surface}}=L\times\chi^{\text{code}}
=L\times\frac{1}{V}\tilde\chi^{\text{code}}
=\frac{1}{A}\tilde\chi^{\text{code}}
,
\end{equation}
i.e. the unnormalized $\tilde\chi^{\text{code}}$ is normalized with
the surface area $A$ of the supercell, giving thus a surface nonlinear
susceptibility, $\chi^{\text{surface}}$.  
For the $\chi^{\text{surface}}$ we take the values given in
\verb=integrands.f90=, and multiply them by
\begin{align}\label{units.1}
\chi^{\text{surface}}[10^6\times\text{pm}^2/\text{V}]=
\chi^{\text{code}}\times \text{acellz}\times52.9177249/10^6
,
\end{align}
then, the 
units are in ($10^6$ pm$^2$/V). 
The 52.9177249 factor is the conversion from Bohr to pm and
divide by 10$^6$ to get the correct scale.

The units  for $\chi^{\text{bulk}}$ must be
\begin{align}\label{units.2}
\chi^{\text{bulk}}=\frac{\text{m}}{\text{V}} 
,
\end{align} 
The units comes from the basic definition of $\chi$ as follows
\begin{equation}\label{u.1}
\bfP^{\text{bulk}}=\chi^{\text{bulk}}\bfE\bfE
,
\end{equation}
where the units of the bulk polarization $\bfP^{\text{bulk}}$
 and electric field $\bfE$
are the same and given by V/m. Therefore, the units of 
$[\chi^{\text{bulk}}]=$m/V.
On the other hand,
the surface polarization $\bfP^s$ is equal to the integration along the surface of      
the bulk polarization, i.e.                                                      
\begin{equation}\label{u.2} 
P^s=\int dz \bfP^{\text{bulk}}\sim d \bfP^{\text{bulk}}                                             
,
\end{equation} 
where $d$ is some distance. Then,                                                  
\begin{equation}\label{u.3}
\bfP^s=d \chi^{\text{bulk}}\bfE\bfE                                             
=\chi^{\text{surface}}\bfE\bfE                                             
,
\end{equation}
where $\chi^s=d\chi^{\text{bulk}}$,
thus
$[\chi^s]=d[\chi^{\text{bulk}}]=d\times$m/V=m$^2$/V.                                                             

\textcolor{red}{Remark:}  
Above expressions with $\calv^{\gS,a}_{nm}$,
are coded under response 44 ($1\go$) and 45 ($2\go$), and with
${\cal R}^{a}_{nm}\to r^a_{nm}$,
under response 21 ($1\go$) and 22 ($2\go$), in 
\verb=integrands.f90=. 
The expressions with $\calv^{\gS,a}_{nm}$ are those of 
S. Andreson et al.\cite{andersonPRB15} 

\textcolor{red}{Remark:} We mention that above expressions with 
${\cal R}^{\rma,\ell}_{nm}\to r^\rma_{nm}$, are coded in
\verb=integrands.f90=, instead of Eq. 40 and 41 of Cabellos et
al.\cite{cabellosPRB09}, which were derived by using Eq. 19 of Aversa
and Sipe.\cite{aversaPRB95} To obtain above equations, we started
from Eq. 18 of Aversa and Sipe,\cite{aversaPRB95} which has the
advantage that applying the layer-by-layer formalism is very
transparent and straightforward. This coding is what constitutes the
{\it Length}-gauge implementation in \tiniba, which is, within a very
small numerical difference, equal to the {\it Velocity}-gauge
implementation of Eq. 35 of Cabellos et al.\cite{cabellosPRB09}, also
in \tiniba. 
{\color{red} THE SPIN FACTOR IS PUT IN
}\verb=file_control.f90=. 
If there is no spin-orbit interaction the 
factor \Verb+spin_factor=2+.  
If there is spin-orbit interaction the 
factor \Verb+spin_factor=1+. The final result is multiplied by  
the \verb=spin_factor= variable. So above expressions are not
multiplied by the spin degeneracy, the code multiplies them.   
